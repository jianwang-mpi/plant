\documentclass[a4paper]{ctexart} %CTEX报告文章格式
\usepackage[top=3cm,bottom=2cm,left=2cm,right=2cm]{geometry} % 页边距
\usepackage{amsthm}
\usepackage{graphicx} %图片
\usepackage{indentfirst}
\setlength{\parindent}{2em}
\title{深度学习在植物物种识别中的应用}
\author{王健$^1$}
\begin{document}
	\maketitle
	\begin{center}
		(1 中国科学院软件研究所,北京 100190, yt4766269@126.com)
	\end{center}
\textbf{关键词}:深度学习,计算机视觉,物种识别

\textbf{摘要}\quad 为了保护物种多样性,了解一个地区的植物物种组成是非常重要的。但是传统的植物物种判别手段非常复杂和耗时,并且要求识别人掌握全面的植物知识。而如今,物种自动识别正引起人们的兴趣。我们可以通过照相机、手机等设备进行图像采集,然后调用物种判别算法,对采集到的图像数据进行物种识别,最终得到具体物种信息。而在上述流程中,物种判别算法起着至关重要的作用。物种判别算法的准确率影响着物种自动识别的好坏。因此
\end{document}