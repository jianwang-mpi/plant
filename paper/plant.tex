\documentclass[a4paper]{ctexart} %CTEX报告文章格式
\usepackage[top=3cm,bottom=2cm,left=2cm,right=2cm]{geometry} % 页边距
\usepackage{amsthm}

\usepackage{graphics} %图片
\usepackage{indentfirst}
\usepackage[hyperref=true,backend=biber,sorting=none,backref=true]{biblatex}
\addbibresource{plant-ref.bib}
\setlength{\parindent}{2em}
\title{深度学习在植物识别中的应用}
\author{王健$^1$}

\begin{document}
	\maketitle
	\begin{center}
		(1 中国科学院软件研究所,北京 100190, yt4766269@126.com)
	\end{center}
\begin{center}
	\textbf{关键词}:深度学习,计算机视觉,植物,叶片
\end{center}


\textbf{摘要}\quad 本文将图像识别研究的最新进展应用于植物图像识别,使用深度学习算法进行叶片分类研究,探究了三种流行的卷积神经网络模型在植物叶片分类上的效果,在经过25个训练循环之后,三种神经网络模型在北美184种植物叶片数据集上均取得了较好的分类效果。

\section*{引言}
为了保护物种多样性,了解一个地区的植物物种组成是非常重要的。但是传统的植物物种判别手段不仅非常复杂和耗时,而且要求识别人掌握全面的植物知识。而如今,物种自动识别方向的研究正引起人们的兴趣。我们可以通过照相机、手机等设备进行图像采集,然后调用物种判别算法,对采集到的图像数据进行分析,最终得到具体物种信息。而在上述流程中,物种判别算法的准确率影响着物种自动识别的好坏,因此物种判别算法起着至关重要的作用。

而随着深度学习理论的发展与计算机视觉技术的进步,近年来涌现出了一大批优秀的深度学习模型,其中包括Vgg\parencite{vgg},Resnet\parencite{resnet},Densenet\parencite{densenet}等,这些深度学习模型在Imagenet\parencite{imagenet}的1000种图像分类的数据集上取得了很好的结果。而植物图像分类问题与Imagenet图像分类问题类似,均可以用深度学习模型很好的解决。因此,本文将如上三种深度学习模型应用于植物图像识别,进行植物叶片照片的分类研究。

\section*{深度学习与图像识别}
2006年,Hinton提出了深度学习。之后深度学习在诸多领域取得了巨大成功,受到广泛关注。神经网络能够重新焕发青春的原因有几个方面:首先,大规模训练数据的出现在很大程度上缓解了训练过拟合的问题。例如,ImageNet训练集拥有上百万个有标注的图像。其次,计算机硬件的飞速发展为其提供了强大的计算能力,一个GPU芯片可以集成上千个核。这使得训练大规模神经网络成为可能。第三,神经网络的模型设计和训练方法都取得了长足的进步。

而深度学习在物体识别中有非常重要的应用,其中最重要的进展体现在ImageNet ILSVRC挑战中的图像分类任务。传统计算机视觉方法在此测试集上最低的错误率是26.172\%。2012年,Hinton的研究小组利用卷积网络把错误率降到了15.315\%。在此之后,不断有新的网络结构被提出出来,网络的深度也不断增加,2014年,Andrew Zisserman等人提出了VGG全卷积网络,在Imagenet数据集上取得了6.8\%的top-5错误率。2015年,微软亚洲研究院的何凯明等人提出了Resnet,将网络层数增加到152层,在Imagenet数据集上取得了3.57\%的top-5错误率。2016年,康奈尔大学的Kilian Q. Weinberger等人,开发出Densenet网络结构,取得了5.3\%的top-5错误率,同时大大加快的网络训练的速度。

虽然深度学习在ImageNet上取得了巨大成功,但是很多应用的训练集是较小的,在此种情况下,可以将ImageNet上训练得到的模型作为起点,利用目标训练集和反向传播对其进行继续训练,将模型适应到特定的应用。此时ImageNet起到预训练的作用。

本文使用了以上三种优秀的神经网络结构,采用了Leafsnap数据集,利用植物的叶片照片进行植物图像分类的研究。

\section*{相关工作}
近年来,有一定数量的文章利用植物叶片图像数据训练分类器进行分类的算法
\section*{机器学习模型}

\section*{实验部分}
\subsection*{数据集}
本文使用了Leafsnap\parencite{leafsnap}数据集,数据集中包括美国北部184种植物叶子图片,共计23,147张图片,其中部分图片如图1所示例。这些数据图片在光照,阴影,清晰度等方面变化很大。该数据集可以通过如下网站下载得到:http://leafsnap.com/dataset/
\begin{figure*}[htbp]
	\centering
	\includegraphics[width=0.75\textwidth]{img1.png}
	\caption{数据集原图}
	\label{figure}
\end{figure*}


在获得数据集之后,需要对原始图片进行预处理,首先将所有图片按照3:1的比例生成训练集与测试集,对于所有图片,将其从图像中间裁剪至256像素×256像素大小,然后进一步随机裁剪至224×224像素大小,然后将图像随机水平翻转,最后将图像归一化,归一化后的图片如图2所示。

\begin{figure*}[htbp]
	\centering
	\includegraphics[width=0.75\textwidth]{img2.png}
	\caption{预处理后图片}
	\label{figure}
\end{figure*}

\subsection*{训练步骤}
在数据预处理完成和选定模型完成之后,可以进行训练参数的设定,这里,我们将三种模型的参数统一设定如下:
\begin{itemize}
	\item 学习速率:0.001
	\item 动量:0.9
	\item batch\_size:4
\end{itemize}
同时,在训练过程中采用了已经预先在Imagenet数据集上训练好的网络参数数据作为预训练的结果。

\subsection*{训练结果}
vgg19在该数据集上获得了92.54\%的准确率,训练的loss曲线如图3所示:

\begin{figure*}[htbp]
	\centering
	\includegraphics[width=0.75\textwidth]{vgg.png}
	\caption{vgg训练曲线}
	\label{figure}
\end{figure*}

Resnet在该数据集上获得了93.25\%的准确率,训练时的loss曲线如图4所示:

\begin{figure*}[htbp]
	\centering
	\includegraphics[width=0.75\textwidth]{resnet.png}
	\caption{Resnet训练曲线}
	\label{figure}
\end{figure*}

densenet在该数据集上获得了93.85\%的准确率,训练时的loss曲线如图5所示:

\begin{figure*}[htbp]
	\centering
	\includegraphics[width=0.75\textwidth]{densenet.png}
	\caption{Densenet训练曲线}
	\label{figure}
\end{figure*}
由上可见,三种模型对植物叶片图像分类的任务均有比较好的效果,而其中Densenet和Resnet训练的效果稍好,并且训练收敛的速度较快。

\subsection*{结果分析}

取densenet的数据进行结果分析,错误率较高的几个类别分别为:
\begin{center}
	\begin{tabular}{|c|c|}
		\hline 植物名称&错误率\\
		\hline prunus\_sargentii&27.4\%\\
		\hline prunus\_virginiana&23\%\\
		\hline quercus\_muehlenbergii&35\%\\
		\hline pinus\_virginiana&55\%\\
		\hline magnolia\_stellata&50\%\\
		\hline
	\end{tabular}
\end{center}
这些类别的叶片形状大致相似,比较难以区分,如图6所示:
	\begin{figure*}[htbp]
		\centering
		\includegraphics[width=0.75\textwidth]{img3.png}
		\caption{错误率较高的几个类别}
		\label{figure}
	\end{figure*}

观察图6,也有可能是数据集存在一些错误导致识别错误率较高。


\section*{结论}
我们在这篇论文中实现了三种深度学习模型,在已有的叶片数据集上进行了训练,显示出较好的结果,这些模型均可以用于植物图像识别的算法。但是这些算法依赖于大量的数据集,对植物叶片数据进行采集和人工分辨也需要耗费相当大的精力。另外,这些算法依赖于服务器上GPU的大规模运算,如果能将其适配于手持设备(如手机)上,则能够更加便于工作人员进行物种的调查采样。



\printbibliography[heading=none]
\end{document}