\documentclass[a4paper]{ctexart} %CTEX报告文章格式
\usepackage[top=3cm,bottom=2cm,left=2cm,right=2cm]{geometry} % 页边距
\usepackage{amsthm}
\usepackage{graphicx} %图片
\usepackage{indentfirst}
\usepackage[hyperref=true,backend=biber,sorting=none,backref=true]{biblatex}
\addbibresource{plant-ref.bib}
\setlength{\parindent}{2em}
\title{深度学习在植物识别中的应用}
\author{王健$^1$}
\begin{document}
	\maketitle
	\begin{center}
		(1 中国科学院软件研究所,北京 100190, yt4766269@126.com)
	\end{center}
\textbf{关键词}:深度学习,计算机视觉,植物,叶片

\textbf{摘要}\quad 本文将图像识别研究的最新进展应用于植物识别,使用深度学习算法进行叶片分类研究,探究了三种流行的卷积神经网络模型在植物叶片分类上的效果,在经过24个训练循环之后,三种神经网络模型在北美184种植物叶片数据集上均取得了较好的分类效果。

\textbf{引言}\quad 为了保护物种多样性,了解一个地区的植物物种组成是非常重要的。但是传统的植物物种判别手段不仅非常复杂和耗时,而且要求识别人掌握全面的植物知识。而如今,物种自动识别方向的研究正引起人们的兴趣。我们可以通过照相机、手机等设备进行图像采集,然后调用物种判别算法,对采集到的图像数据进行分析,最终得到具体物种信息。而在上述流程中,物种判别算法的准确率影响着物种自动识别的好坏,因此物种判别算法起着至关重要的作用。

而随着深度学习理论的发展与计算机视觉技术的进步,近年来涌现出了一大批优秀的深度学习模型,其中包括Inception~\parencite{A01},Resnet,Densenet等,这些深度学习模型在Imagenet的1000种图像分类的数据集上取得了很好的结果,

\printbibliography
\end{document}